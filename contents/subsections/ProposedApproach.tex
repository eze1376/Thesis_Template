% Chapter 1

\chapter{روش پیشنهادی}

\section{مقدمه}

\subsection{نمونه جدول فارسی}

\begin{table}
	\renewcommand{\arraystretch}{2}
	\fontsize{12}{6}
	\begin{center}
		\caption{زمان تشخیص و ردیابی توپ بر روی دو رایانه قوی و ضعیف}
		\label{table1:StrongWeakSystemBallDetectionTrackingSpeed}
		\begin{tabular}{|c|c|c|}
			\hline
			\multicolumn{1}{|c|}{\bfseries نمونه اعداد} & \multicolumn{1}{c|}{\bfseries \begin{tabular}{@{}c@{}c@{}c@{}} \\ نمونه اول \\ ‌\end{tabular}  
			} & \multicolumn{1}{c|}{\bfseries \begin{tabular}{@{}c@{}c@{}c@{}} \\ نمونه \\
					دوم\\ ‌\end{tabular}} \\
			\hline
			\begin{tabular}{@{}c@{}c@{}}انگلیسی\end{tabular} & ‍\lr{25.7} & \lr{4.74}\\
			\hline
			\begin{tabular}{@{}c@{}c@{}} فارسی \end{tabular} & ۴٫۷۴ & ۲۵٫۷\\
			\hline
			\begin{tabular}{@{}c@{}c@{}}  ریاضی \end{tabular} & $4.75$ & $25.7$\\
			\hline
		\end{tabular}
	\end{center}
\end{table}

\subsection{نمونه فرمول ریاضی}

\begin{equation}
	C = 2 \pi r
	\label{eq:CircumferenceOfCircle}
\end{equation}

\section{جمع‌بندی}
